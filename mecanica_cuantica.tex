\documentclass{article}

\usepackage{amsmath}
\usepackage[utf8]{inputenc}
\usepackage[spanish]{babel}
\decimalpoint
\usepackage{fullpage}
\usepackage[pdffitwindow,colorlinks,citecolor={red},linkcolor={blue}]{hyperref}

\title{Mecánica Cuántica: Las Matemáticas del Micro-Universo}
\author{Dr. en C. Reinaldo Arturo Zapata Peña}
\date{}

\begin{document}
\maketitle

La mecánica cuántica surge a partir de la necesidad de describir procesos de la
naturaleza que no tienen explicación en el marco teórico de la física clásica.
% 
En los primeros años del siglo XX la comunidad científica tenía acumulado una
serie de problemas fundamentales que no podían resolver, no debido a la
complejidad de los mismos, sino debido a la falta de una teoría que describiera
su naturaleza. 
% 
Mas allá de esto, las posibles soluciones postuladas a dichos problemas caían
en errores y la teoría física, que se consideraba completa hasta ese entonces,
fue cuestionada desde sus fundamentos.
% 
La búsqueda a la solución a estos problemas se intensificó durante las primeras
décadas del siglo XX por la comunidad científica abarcando figuras importantes
de la física y las matemáticas. La búsqueda partió del hecho de que la
mecánica newtoniana no es directamente aplicable al estudio del átomo y,
% 
de esta manera, la mecánica cuántica florece como una nueva teoría que permite
la descripción de sistemas cuánticos microscópicos, como los átomos y
moléculas, y macroscópicos como los superconductores y láseres, entre otros.

Históricamente el problema del \emph{cuerpo negro} tiene gran importancia pues
fue el primero cuya descripción reveló la necesidad del desarrollo de esta
nueva teoría. 
% 
Se puede definir como cuerpo negro ideal, que se encuentra a una temperatura
$T$ (absoluta), a un sistema que se caracteriza por absorber prácticamente toda
la radiación incidente sobre él \cite{planck2013theory}.
% 
De esta forma, el modelado típico de un cuerpo negro consiste en una cavidad
hueca metálica a la que se le hace una una pequeña perforación; toda la
radiación que entra es absorbida tras numerosas reflexiones en el interior.
% 
Acorde con la teoría electromagnética clásica la radiación que escapa es la
correspondiente a la de un cuerpo negro dicha temperatura que además tiene
componentes en todo el rango del espectro electromagnético. De esta manera la
energía radiada por unidad de área en todas las longitudes de onda es
\begin{equation}\label{eq:radiacion}
u = \int _{0}^{\infty} \rho (\omega, T) d\omega,
\end{equation}
donde 
% 
\begin{equation}
\rho(\omega, T) = \frac{\omega^2}{\pi^{2}C^{3}}kT
\end{equation}
% 
es la densidad espectral de la radiación emitida. La
Eq. \eqref{eq:radiacion} indica entonces que un cuerpo negro podría liberar una
cantidad infinita de energía, contradiciendo el principio de conservación de la
energía.

Por otro lado, la Ley de Stefan–Boltzmann describe la potencia radiada por un
cuerpo negro en términos de su temperatura estableciendo que la energía total
radiada por unidad de área a lo largo de todas las longitudes de onda por
unidad de tiempo es directamente proporcional a la cuarta potencia de la
temperatura termodinámica de la temperatura del cuerpo negro
\begin{equation}\label{eq:ley-stefan-boltzmann}
u = aT^{4},
\end{equation}
donde $a =4 \sigma/c$, $\sigma$ es la constante de Stefan–Boltzmann y $c$ es la
velocidad de la luz. 


% The Stefan–Boltzmann law describes the power radiated from a black body in terms of its temperature. Specifically, the Stefan–Boltzmann law states that the total energy radiated per unit surface area of a black body across all wavelengths per unit time 

% ⋆ j^{\star} (also known as the black-body radiant emittance) is directly proportional to the fourth power of the black body's thermodynamic temperature T:



\bibliography{mecanica_cuantica.bib}
\bibliographystyle{plain}

\end{document}




