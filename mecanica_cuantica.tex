\documentclass{article}

\usepackage{amsmath}
\usepackage[utf8]{inputenc}
\usepackage[spanish]{babel}
\decimalpoint
\usepackage{fullpage}
\usepackage[pdffitwindow,colorlinks,citecolor={red},linkcolor={blue}]{hyperref}

\title{Mecánica Cuántica: Las Matemáticas del Micro-Universo}
\author{Dr. en C. Reinaldo Arturo Zapata Peña}
\date{}

\begin{document}
\maketitle

La mecánica cuántica surge a partir de la necesidad de describir procesos de la
naturaleza que no tienen explicación en el marco teórico de la física clásica
siendo una nueva teoría que permite la descripción de sistemas cuánticos
microscópicos, como los átomos y moléculas, y macroscópicos como los
superconductores y láseres, entre otros.
% 
Históricamente el problema del \emph{cuerpo negro} en equilibrio térmico a una
temperatura $T$ tiene gran importancia pues fue el primero cuya descripción
reveló la necesidad del desarrollo de esta nueva teoría.
% 
Se puede definir como cuerpo negro ideal, a un sistema que se caracteriza por
absorber toda la radiación incidente sobre él y se puede modelar mediante una
cavidad hueca metálica a la que se le hace una una pequeña perforación. Toda la
radiación que entra es absorbida tras numerosas reflexiones en el interior y,
además, acorde a la teoría electromagnética clásica, un cuerpo negro en
equilibrio térmico emitirá radiación en todos los rangos de energía
\cite{planck2013theory}.

La potencia radiada en todas direcciones por unidad de área de la superficie de
un cuerpo negro en función de su temperatura absoluta $T$ es descrita por la
ley de Stefan-Boltzmann
% 
\begin{equation}\label{eq:ley-stefan-boltzmann}
j = \sigma T^{4},
\end{equation}
% 
donde $\sigma$ es la constante de Stefan–Boltzmann; entonces la densidad de
energía de la radiación se puede obtener mediante 
%
\begin{equation}
u= aT^{4},
\end{equation}
donde $a = 4\sigma/c$. Haciendo un análisis espectral de $u$ tenemos que posee
componentes de todas las frecuencias, entonces, sumando las contribuciones de
toas ellas tenemos que 
% 
\begin{equation}
u = \int_{0}^{\infty} \rho(\omega, T)d\omega,
\end{equation}
% 
donde $\rho(\omega, T)$ es la densidad espectral de radiación.




% La Ley de Rayleigh–Jeans describe de forma clásica la radiancia\footnote{Flujo
% de radiación en una superficie por unidad de ángulo sólido por unidad de área
% proyectada.}, $B_{\lambda}(T)$ en función de la longitud de onda $\lambda$,
% emitida por un cuerpo negro que se encuentra a una temperatura $T$
% % 
% \begin{equation}
% B_{\lambda} (T) = \frac{2ck_{B}T}{\lambda^{4}},
% \end{equation}
% % 
% o bien, en función de la frecuencia
% \begin{equation}
% B_{\lambda} (T) = \frac{2\nu^{2}k_{B}T}{c^{2}},
% \end{equation}
% % 
% donde $c$ es la velocidad de la luz, $K_{B}$ es la constante de Boltzmann, 

\bibliography{mecanica_cuantica.bib}
\bibliographystyle{plain}

\end{document}




